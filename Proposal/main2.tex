\documentclass[12pt,a4paper]{report}
\usepackage[a4paper,margin=1in]{geometry}
\usepackage{graphicx}
\usepackage{setspace}
\usepackage{array}
\usepackage{ragged2e}
\usepackage{parskip}
\usepackage{tikz}
\usetikzlibrary{calc}
\usepackage{everypage}
\usepackage{float}
\usepackage{longtable}
\usepackage{hyperref}
\usepackage{enumitem}
\usepackage{fancyhdr}

\AddEverypageHook{%
    \begin{tikzpicture}[remember picture,overlay]
        \draw[line width=1pt]
            ($(current page.north west) + (1cm,-1cm)$)
            rectangle
            ($(current page.south east) + (-1cm,1cm)$);
    \end{tikzpicture}
}

\setcounter{secnumdepth}{4}
\setcounter{tocdepth}{4}

\pagestyle{fancy}
\fancyhf{}
\fancyhead[L]{\leftmark}
\fancyhead[R]{CattleCare Pro SRS}
\fancyfoot[C]{\thepage}
\renewcommand{\headrulewidth}{0.4pt}

\begin{document}

% ================= TITLE PAGE =================
\begin{titlepage}
\centering
\vspace*{1cm}

{\Huge \textbf{Software Requirements Specification}}\\[0.5cm]
{\LARGE \textbf{CattleCare Pro}}\\[2cm]

{\large IEEE Std 830-1998 Compliant Document}\\[2cm]

\textbf{Department of Computer Science}\\
Namal University Mianwali\\[1.9cm]

\textbf{Prepared By:}\\[0.3cm]
\begin{tabular}{ll}
Muhammad Ali & NUM-BSCS-2024-46\\
Muhammad Ahmad & NUM-BSCS-2024-45\\
Aliya Ashraf & NUM-BSCS-2024-08
\end{tabular}

\vfill

\textbf{Submission Date:} \today

\textbf{Document Status:} Final Draft\\
\textbf{Submitted to:} Ms. Asiya Batool
\end{titlepage}

\tableofcontents
\newpage

\listoftables
\newpage

% ======================================================
\chapter{Introduction}
\begin{spacing}{1.5}

\section{Purpose}
\subsection{Document Purpose}
This Software Requirements Specification (SRS) document provides a complete and precise description of the requirements for the CattleCare Pro livestock monitoring and management system. This document has been prepared in strict accordance with IEEE Std 830-1998, "IEEE Recommended Practice for Software Requirements Specifications," to ensure clarity, completeness, and verifiability of all stated requirements.

The primary purposes of this document are to:
\begin{itemize}
\item Define all functional and non-functional requirements for the CattleCare Pro system
\item Establish a contractual foundation between stakeholders and developers
\item Provide a reference for validation and verification activities
\item Serve as a basis for cost estimation and project planning
\item Guide system design and implementation decisions
\end{itemize}

\subsection{Intended Audience and Reading Suggestions}
This SRS is intended for multiple stakeholders, each with specific interests:

\subsubsection{Software Developers and Engineers}
Developers will use this document as the primary reference for understanding what functionality must be implemented. Section 3 (Specific Requirements) contains detailed technical specifications that should be consulted throughout the development lifecycle.

\subsubsection{Project Managers and Supervisors}
Project managers will use this document to plan resources, estimate costs, and schedule development milestones. The product scope (Section 1.3) and constraints (Section 2.6) are particularly relevant for project planning purposes.

\subsubsection{Quality Assurance and Testing Teams}
QA teams will use this SRS to develop test plans, test cases, and validation criteria. Every requirement in this document is written to be verifiable, meaning that test procedures can be designed to objectively determine whether each requirement has been satisfied.

\subsubsection{Academic Evaluators}
Faculty members and academic evaluators will use this document to assess the completeness and quality of the requirements engineering process, as well as compliance with IEEE standards and software engineering best practices.

\subsubsection{Farm Owners and Stakeholders}
End users and farm management personnel will use sections 1 and 2 to understand the system's capabilities, limitations, and expected benefits.

\subsection{Product Scope}
\subsubsection{Product Name and Identity}
The software product specified in this document is named \textbf{CattleCare Pro}, a comprehensive livestock monitoring and management system designed specifically for cattle farming operations.

\subsubsection{Product Benefits and Objectives}
CattleCare Pro aims to revolutionize cattle farming through technology-enabled monitoring and data-driven decision making. The primary benefits include:

\paragraph{Enhanced Animal Welfare}
By providing real-time monitoring of cattle location and activity patterns, the system enables farmers to quickly identify animals in distress, and respond promptly to emergency situations.

\paragraph{Operational Efficiency}
The system automates many time-consuming manual tasks such as counting animals, monitoring their locations, and tracking their movements, freeing farm workers to focus on higher-value activities.

\paragraph{Data-Driven Decision Making}
Historical data analytics and trend analysis empower farm managers to make informed decisions about feeding schedules, breeding programs, and veterinary interventions based on objective data.

\paragraph{Theft Prevention and Security}
GPS tracking and geofencing capabilities help prevent cattle theft, a significant problem in many agricultural regions. The system can alert owners immediately when animals move beyond designated boundaries.

\subsubsection{Product Scope Boundaries}
\paragraph{Within Scope}
\begin{itemize}
\item Real-time GPS tracking of individual cattle
\item Activity monitoring and behavior pattern analysis
\item Automated anomaly detection and alert generation
\item Cloud-based data storage and management
\item Mobile application for Android platforms
\item Analytics dashboards using Power BI
\item User authentication and access control
\item Historical data analysis and trend identification
\end{itemize}

\paragraph{Outside Scope}
\begin{itemize}
\item Veterinary diagnosis or medical treatment recommendations
\item Automated feeding systems or equipment control
\item Financial management or accounting functions
\item Integration with existing legacy farm management systems (initial version)
\end{itemize}

\section{Document Conventions}
\subsection{Requirement Priority and Necessity}
This SRS follows IEEE recommended conventions for expressing requirements:

\subsubsection{Mandatory Requirements}
Requirements expressed using the keyword \textbf{shall} are mandatory and must be implemented exactly as specified. For example: "The system shall authenticate users before granting access."

\subsubsection{Desirable Requirements}
Requirements expressed using \textbf{should} are highly desirable but not absolutely mandatory. For example: "The system should provide data export capabilities in multiple formats."

\subsubsection{Optional Requirements}
Requirements expressed using \textbf{may} are optional features that could enhance the system but are not required for basic operation.

\subsection{Typographical Conventions}
\begin{itemize}
\item \textbf{Bold text} is used for requirement keywords (shall, should, may) and emphasis
\item \textit{Italic text} is used for document references and technical terminology
\item \texttt{Monospace font} is used for system messages and technical identifiers
\item ALL CAPS is used for acronyms and abbreviations
\end{itemize}

\subsection{Requirement Labeling Convention}
Each specific requirement is assigned a unique identifier:
\begin{itemize}
\item FR-XX: Functional Requirements
\item NFR-XX: Non-Functional Requirements
\item IR-XX: Interface Requirements
\item DR-XX: Design Constraint Requirements
\end{itemize}

\section{Definitions, Acronyms, and Abbreviations}
\subsection{Definitions}
\paragraph{Anomaly}
Any deviation from normal cattle behavior patterns as determined by the system's algorithms.

\paragraph{Cloud Platform}
Remote server infrastructure accessed via the internet that provides data storage and processing capabilities.

\paragraph{GPS Collar}
GPS-enabled hardware device attached to cattle that collects location and movement data.

\paragraph{Dashboard}
Visual interface displaying key metrics, charts, and summary information about cattle status.

\paragraph{Farm Administrator}
Primary system user with full access privileges, typically the farm owner or manager.

\paragraph{Geofence}
Virtual boundary defined by GPS coordinates that creates a designated area.

\paragraph{Real-time}
Information processing and display occurring with minimal delay (typically less than 5 seconds).

\subsection{Acronyms}
\begin{longtable}{|p{3cm}|p{10cm}|}
\hline
\textbf{Acronym} & \textbf{Definition} \\ \hline
API & Application Programming Interface \\ \hline
GPS & Global Positioning System \\ \hline
HTTPS & Hypertext Transfer Protocol Secure \\ \hline
IEEE & Institute of Electrical and Electronics Engineers \\ \hline
IoT & Internet of Things \\ \hline
JSON & JavaScript Object Notation \\ \hline
SRS & Software Requirements Specification \\ \hline
UI & User Interface \\ \hline
UX & User Experience \\ \hline
\end{longtable}

\section{References}
\begin{enumerate}
\item IEEE Std 830-1998, IEEE Recommended Practice for Software Requirements Specifications
\item Course Material — Software Engineering, Namal University, Fall 2025
\end{enumerate}

\section{Overview}
The remainder of this SRS is organized as follows:

\textbf{Chapter 2: Overall Description} provides high-level context including product perspective, functions, user characteristics, constraints, and assumptions.

\textbf{Chapter 3: Specific Requirements} contains detailed functional and non-functional requirements with unique identifiers.

\textbf{Chapter 4: System Models} presents visual representations including use case diagrams, and context diagrams.

\textbf{Chapter 5: Appendices} contains supporting information and additional documentation.

\end{spacing}

% ======================================================
\chapter{Overall Description}
\begin{spacing}{1.5}

\section{Product Perspective}
CattleCare Pro is a new, self-contained software system designed specifically for livestock monitoring and management. The system integrates multiple technology components:

\subsection{Hardware Components}
\begin{itemize}
\item \textbf{GPS Collar Devices:} Wearable IoT devices attached to cattle that collect location and movement data
\item \textbf{Mobile Devices:} Android smartphones and tablets used by farm personnel
\item \textbf{Cameras:} Visual monitoring equipment for enhanced surveillance
\end{itemize}

\subsection{Software Components}
\begin{itemize}
\item \textbf{Cloud Backend Services:} Server applications that process data and manage storage
\item \textbf{Mobile Application:} Native Android app providing user interface
\item \textbf{Analytics Platform:} Power BI integration for data visualization
\end{itemize}

\subsection{Communication Infrastructure}
\begin{itemize}
\item \textbf{Cellular Networks:} 4G connectivity for data transmission
\item \textbf{WiFi Networks:} Local wireless connectivity for mobile devices
\item \textbf{Internet Connectivity:} Broadband connection for cloud service access
\end{itemize}

\section{Product Functions}
The major functions of CattleCare Pro include:

\subsection{Real-Time Cattle Tracking}
The system shall continuously monitor and display the current location of all cattle equipped with GPS collars, enabling:
\begin{itemize}
\item View cattle locations on interactive maps
\item Track movement patterns throughout the day
\item Identify exact positions during emergencies
\item Verify animals remain within designated areas
\end{itemize}

\subsection{Activity Monitoring and Analysis}
The system shall analyze movement data to understand cattle behavior:
\begin{itemize}
\item Detect periods of activity versus rest
\item Identify unusual behavior patterns
\item Track grazing patterns and feeding behavior
\item Monitor reproductive behavior indicators
\end{itemize}

\subsection{Anomaly Detection and Alerting}
The system shall automatically identify abnormal situations:
\begin{itemize}
\item Cattle leaving designated areas (geofence violations)
\item Extended periods of inactivity suggesting illness
\item Unusual movement patterns
\item Device malfunctions or connectivity loss
\end{itemize}

\subsection{Data Storage and Management}
All collected data shall be securely stored in cloud infrastructure with:
\begin{itemize}
\item Redundancy to prevent data loss
\item Encryption to protect privacy
\item Retention of historical data for trend analysis
\item Efficient retrieval mechanisms
\end{itemize}

\subsection{Analytics and Reporting}
The system shall provide sophisticated analysis capabilities:
\begin{itemize}
\item Historical trend analysis using Power BI
\item Comparative analysis between animals or groups
\item Customizable reports for different needs
\item Visual dashboards showing key performance indicators
\item Export capabilities for external analysis
\end{itemize}

\subsection{User Management}
The system shall provide secure authentication and authorization:
\begin{itemize}
\item User account creation and management
\item Role-based access control
\item Password management and recovery
\end{itemize}

\section{User Classes and Characteristics}

\subsection{Farm Owner / Administrator}
\subsubsection{Characteristics}
\begin{itemize}
\item \textbf{Technical Skill Level:} Basic to intermediate
\item \textbf{Usage Frequency:} Daily, multiple times per day
\item \textbf{Primary Device:} Mobile phone
\end{itemize}

\subsubsection{Responsibilities}
\begin{itemize}
\item Monitor overall herd health and status
\item Make strategic decisions based on system data
\item Respond to critical alerts
\item Configure system settings
\end{itemize}

\subsection{Farm Worker}
\subsubsection{Characteristics}
\begin{itemize}
\item \textbf{Technical Skill Level:} Basic
\item \textbf{Usage Frequency:} Daily during work hours
\item \textbf{Role:} Indirect - provides information to Farm Owner
\end{itemize}

\subsubsection{Responsibilities}
\begin{itemize}
\item Observe cattle during field rounds
\item Report observations to administrator
\item Assist in physical cattle management
\end{itemize}

\subsection{System Administrator}
\subsubsection{Characteristics}
\begin{itemize}
\item \textbf{Technical Skill Level:} Advanced
\item \textbf{Usage Frequency:} As needed for maintenance
\end{itemize}

\subsubsection{Responsibilities}
\begin{itemize}
\item System installation and configuration
\item User account management
\item Database maintenance and backups
\item Security monitoring
\item Performance optimization
\item Technical support
\end{itemize}

\begin{table}[H]
\centering
\caption{User Class Comparison}
\begin{tabular}{|p{3cm}|p{2.5cm}|p{3cm}|p{4cm}|}
\hline
\textbf{User Class} & \textbf{Skill Level} & \textbf{Primary Device} & \textbf{Key Requirements} \\ \hline
Farm Owner/Admin & Basic-Intermediate & Mobile & Easy interface, comprehensive data \\ \hline
Farm Worker & Basic & None (indirect) & Manual reporting to owner \\ \hline
System Admin & Advanced & Desktop & Full system access, technical tools \\ \hline
\end{tabular}
\end{table}

\section{Operating Environment}

\subsection{Hardware Environment}
\subsubsection{Collar Devices}
\begin{itemize}
\item Rugged, weatherproof construction (IP67 rating)
\item Battery life: minimum 7 days
\item GPS accuracy: ±5 meters
\item Operating temperature: -20°C to +50°C
\item Weight: Maximum 500 grams
\item Cellular connectivity: 4G with 3G fallback 
\end{itemize}

\subsubsection{Mobile Devices}
\begin{itemize}
\item Android devices
\item Cellular or WiFi connectivity required
\item Minimum 2GB RAM, 32GB storage
\end{itemize}

\subsection{Software Environment}
\subsubsection{Cloud Platform}
\begin{itemize}
\item Microsoft Azure or Amazon Web Services
\item Scalable compute and storage resources
\item IoT Hub or equivalent service
\item Database service (SQL or NoSQL)
\end{itemize}

\subsubsection{Analytics Platform}
\begin{itemize}
\item Microsoft Power BI Desktop and Service
\item Power BI Mobile applications
\item Data gateway for cloud connectivity
\end{itemize}

\subsection{Network Environment}
\begin{itemize}
\item \textbf{Cellular:} 4G/3G coverage in grazing areas
\item \textbf{Mobile:} Cellular data or farm WiFi
\end{itemize}

\section{Design and Implementation Constraints}

\subsection{Regulatory Compliance}
\begin{itemize}
\item Compliance with local data protection laws
\item Animal welfare monitoring requirements
\item Wireless communication regulations
\end{itemize}

\subsection{Technology Constraints}
\begin{itemize}
\item \textbf{Battery Life:} GPS readings must balance accuracy with power consumption
\item \textbf{Network Coverage:} Limited cellular coverage in rural areas
\item \textbf{GPS Accuracy:} Signal quality affected by terrain and weather
\end{itemize}

\subsection{Development Constraints}
\begin{itemize}
\item Fixed academic semester timeline
\item Development team size: 3 developers
\item Limited budget for cloud services
\item Access to farm environments for testing
\end{itemize}

\subsection{Security Constraints}
\begin{itemize}
\item Multi-factor authentication for admin access
\item End-to-end encryption for sensitive data
\item Secure communication protocols (HTTPS)
\end{itemize}

\section{Assumptions and Dependencies}

\subsection{Assumptions}
\begin{itemize}
\item \textbf{A1:} Adequate cellular network coverage exists in grazing areas
\item \textbf{A2:} Farm owner possesses Android smartphone
\item \textbf{A3:} Reliable internet connectivity available at farm
\item \textbf{A4:} GPS collars can be securely attached to cattle
\item \textbf{A5:} Cloud service providers maintain service levels
\item \textbf{A6:} Users have basic smartphone operation skills
\item \textbf{A7:} Users willing to adopt technology-based practices
\item \textbf{A8:} Collar devices will be regularly maintained
\end{itemize}

\subsection{Dependencies}
\begin{itemize}
\item \textbf{D1:} Cloud platform services availability
\item \textbf{D2:} Cellular network infrastructure
\item \textbf{D3:} GPS satellite system operation
\item \textbf{D4:} Android operating system updates
\item \textbf{D5:} Power BI platform licensing
\item \textbf{D6:} GPS collar device manufacturer support
\item \textbf{D7:} Cellular service provider coverage
\end{itemize}

\end{spacing}

% ======================================================
\chapter{Specific Requirements}
\begin{spacing}{1.5}

\section{Functional Requirements}

\subsection{User Authentication and Authorization}

\paragraph{FR-01: User Account Creation}
The system shall allow authorized administrators to create new user accounts by providing username, email address, password, full name, and role assignment. Account creation shall generate a unique user identifier.

\paragraph{FR-02: Secure Login}
The system shall authenticate users by verifying username/email and password credentials against the encrypted database. Upon successful authentication, the system shall create a secure session token.

\paragraph{FR-03: Password Security}
The system shall enforce password requirements: minimum 8 characters, at least one uppercase letter, one lowercase letter, one number, and one special character. Passwords shall be stored using bcrypt hashing.

\paragraph{FR-04: Password Recovery}
The system shall provide password reset functionality via email verification. Users shall receive a time-limited reset link to set a new password.

\paragraph{FR-05: Session Management}
The system shall automatically terminate inactive user sessions after 15 minutes.

\subsection{Cattle Registration and Management}

\paragraph{FR-06: Add New Cattle Record}
The system shall allow administrators to register new cattle by entering unique identification number (tag ID), collar device ID, breed, date of birth, gender, and optional notes. The system shall validate uniqueness of tag ID and collar ID.

\paragraph{FR-07: Update Cattle Information}
The system shall allow authorized users to modify cattle profile information. All changes shall be logged with timestamp and user identification.

\paragraph{FR-08: Remove Cattle Record}
The system shall allow administrators to deactivate cattle records when animals are sold or deceased. Historical data shall be retained for archived records.

\paragraph{FR-09: Search and Filter Cattle}
The system shall provide search and filter capabilities by tag ID, collar ID, breed, age range, gender, and status. Results shall be displayed in sortable tables.

\subsection{Real-Time Location Tracking}

\paragraph{FR-10: Receive GPS Coordinates}
The system shall continuously receive GPS coordinates (latitude, longitude) from all active collar devices. Data reception shall include timestamp, device ID, and accuracy estimate.

\paragraph{FR-11: Data Transmission Frequency}
GPS collar devices shall transmit location data at configurable intervals, with default setting of every 5 minutes during daytime and every 15 minutes during nighttime.

\paragraph{FR-12: Display Current Locations}
The system shall display current location of all tracked cattle on an interactive map. Each cattle location shall be represented by a marker showing tag ID.

\paragraph{FR-13: Map Interaction Features}
The map interface shall support zoom, pan, marker click for details, and layer controls for geofences.

\paragraph{FR-14: Location Refresh}
The system shall automatically update displayed cattle locations when new GPS data is received without requiring manual refresh.

\paragraph{FR-15: Individual Cattle Tracking}
The system shall allow users to select a specific animal and view its current location, recent movement path, and relevant details.

\paragraph{FR-16: Store Location History}
The system shall store all received GPS coordinates in the cloud database with associated metadata: timestamp, cattle ID, coordinates, accuracy, and battery level.

\paragraph{FR-17: Display Historical Paths}
The system shall allow users to query and display historical movement paths for any cattle over a specified date range. Paths shall be rendered as connected lines on the map.


\subsection{Activity Monitoring}

\paragraph{FR-18: Record Movement Data}
Collar devices shall record accelerometer data to detect animal movement and activity levels. Activity metrics shall be computed and transmitted with location data.

\paragraph{FR-19: Activity Classification}
The system shall classify cattle activity into categories: Active (moving/grazing), Resting (lying down), and Feeding (head-down position).

\paragraph{FR-20: Calculate Daily Activity Metrics}
The system shall compute daily activity summaries for each animal: total active time, and total rest time.

\paragraph{FR-21: Activity Pattern Recognition}
The system shall analyze activity patterns over time to establish baseline behavior profiles and detect deviations from normal patterns.

\paragraph{FR-22: Comparative Activity Analysis}
The system shall allow users to compare activity levels between multiple cattle or between an individual and herd average.

\subsection{Geofence Management}

\paragraph{FR-23: Create Geofence Areas}
The system shall allow administrators to define geofence boundaries by drawing polygons on a map or entering GPS coordinates. Each geofence shall have a unique name and description.

\paragraph{FR-24: Multiple Geofence Support}
The system shall support creation of multiple geofence areas representing different pastures and restricted zones.

\paragraph{FR-25: Modify Geofences}
The system shall allow administrators to edit geofence boundaries and active status without deleting historical violation records.

\paragraph{FR-26: Detect Geofence Violations}
The system shall evaluate each GPS location against all active geofences and detect when cattle enter or exit boundaries. Violations shall be recorded with timestamp.

\paragraph{FR-27: Geofence Status Display}
The system shall display current geofence compliance status for all cattle on the monitoring dashboard.

\subsection{Alert and Notification System}

\paragraph{FR-28: Geofence Violation Alerts}
The system shall automatically generate high-priority alerts when cattle exit designated safe zones. Alerts shall include cattle identification, location, time, and geofence name.

\paragraph{FR-29: Inactivity Alerts}
The system shall generate medium-priority alerts when cattle activity falls below threshold levels for extended periods.

\paragraph{FR-30: Device Malfunction Alerts}
The system shall generate alerts when collar device battery falls below 15.

\paragraph{FR-31: Unusual Activity Alerts}
The system shall generate alerts when cattle exhibit abnormal activity patterns significantly different from established baselines.

\paragraph{FR-32: Push Notifications}
The system shall deliver alert notifications to mobile devices via push notification service within 30 seconds of alert generation.

\paragraph{FR-33: In-App Alert Center}
The mobile application shall maintain an Alert Center displaying all alerts with status (new, acknowledged, resolved).

\paragraph{FR-34: Acknowledge Alerts}
The system shall allow users to acknowledge alerts to indicate awareness. Acknowledged alerts shall be marked with timestamp and user identity.

\paragraph{FR-35: Resolve Alerts}
The system shall allow users to mark alerts as resolved with optional resolution notes.

\paragraph{FR-36: Configure Alert Preferences}
The system shall allow administrators to configure alert thresholds, notification channels, and user assignments.

\subsection{Data Analytics and Reporting}

\paragraph{FR-37: Real-Time Dashboard}
The system shall provide a main dashboard displaying total cattle count, cattle by status, active alerts count, geofence compliance summary, and recent activity highlights.

\paragraph{FR-38: Individual Cattle Dashboard}
The system shall provide detailed dashboard for individual animals showing current location, activity graphs, alert history, and key metrics.

\paragraph{FR-39: Power BI Integration}
The system shall integrate with Microsoft Power BI for advanced data visualization, enabling users to create custom dashboards and reports.

\paragraph{FR-40: Daily Activity Reports}
The system shall generate automated daily reports summarizing cattle locations, activity levels, alerts generated, and any incidents.

\paragraph{FR-41: Scheduled Reports}
The system shall allow administrators to schedule automated report generation and delivery via email at specified intervals (daily, weekly, monthly, yearly).

\paragraph{FR-42: Trend Analysis}
The system shall provide trend analysis tools showing activity patterns, location preferences, and behavior changes over selectable time periods.

\subsection{System Administration}

\paragraph{FR-43: User Account Management}
The system shall allow administrators to create new accounts, modify user details, reset passwords, and deactivate accounts. All actions shall be logged.

\paragraph{FR-44: General System Settings}
The system shall provide configuration interface for organization details, time zone, measurement units, and date/time formats.

\paragraph{FR-45: Device Management}
The system shall allow administrators to register new collar devices, assign devices to cattle, modify device settings, and view device status.

\paragraph{FR-46: Notification Configuration}
The system shall allow configuration of alert thresholds, and notification channels.

\paragraph{FR-47: Data Backup}
The system shall perform automated daily backups of all system data to secure cloud storage with retention period of 90 days.


\section{Non-Functional Requirements}

\subsection{Performance Requirements}

\paragraph{NFR-01: User Interface Response}
The system shall respond to 95\% of user interface interactions within 2 seconds under normal load conditions.

\paragraph{NFR-02: Map Loading Time}
The interactive map with cattle locations shall load and display within 3 seconds for herds up to 500 animals.

\paragraph{NFR-03: Alert Delivery Time}
Critical alerts shall be delivered to user mobile devices within 30 seconds of alert trigger event.

\paragraph{NFR-04: Data Synchronization}
The system shall synchronize new GPS data to the database within 10 seconds of data reception by cloud platform.

\paragraph{NFR-05: Report Generation}
Standard reports shall be generated within 10 seconds for date ranges up to 30 days.

\paragraph{NFR-06: Concurrent GPS Processing}
The system shall process GPS data from at least 1,000 collar devices simultaneously.

\paragraph{NFR-07: Concurrent User Support}
The system shall support at least 50 concurrent users accessing the mobile simultaneously without performance degradation.

\paragraph{NFR-08: Database Query Performance}
Database queries for current cattle status shall execute in under 1 second for herds up to 1,000 animals.

\subsection{Reliability Requirements}

\paragraph{NFR-09: System Uptime}
The system shall maintain 99.9\% uptime excluding planned maintenance.

\paragraph{NFR-10: Data Recovery Time}
In the event of system failure, data recovery and system restoration shall be completed within 1 hour.

\paragraph{NFR-11: Backup Verification}
System backups shall be verified weekly to ensure recoverability. Backup restoration testing shall be performed quarterly.

\paragraph{NFR-12: Error Handling}
The system shall implement comprehensive error handling with graceful degradation and informative error messages to users.

\subsection{Security Requirements}

\paragraph{NFR-13: Data Encryption}
All sensitive data shall be encrypted.

\paragraph{NFR-14: Password Storage}
User passwords shall be hashed using bcrypt algorithm. Plain text passwords shall never be stored.

\paragraph{NFR-15: Session Security}
User sessions shall be secured with randomly generated tokens, automatic timeout after 15 minutes of inactivity, and secure cookie attributes.

\paragraph{NFR-16: Access Control}
The system shall enforce role-based access control, preventing unauthorized users from accessing restricted features or data.

\paragraph{NFR-17: Data Privacy}
The system shall ensure that only authorized users can access cattle and farm data, maintaining confidentiality at all times.

\subsection{Usability Requirements}

\paragraph{NFR-18: Learning Curve}
Users with basic smartphone skills shall be able to perform common tasks (view cattle locations, respond to alerts) within 15 minutes of first use.

\paragraph{NFR-19: Interface Simplicity}
The mobile application interface shall be intuitive and easy to navigate for users with minimal technical knowledge.

\paragraph{NFR-20: Help Documentation}
The system shall provide context-sensitive help and user documentation accessible within the application.

\paragraph{NFR-21: Error Messages}
Error messages shall be clear, specific, and provide guidance on how to resolve the issue.

\paragraph{NFR-22: Consistency}
User interface elements, terminology, and interaction patterns shall be consistent throughout the application.

\subsection{Maintainability Requirements}

\paragraph{NFR-23: Code Documentation}
Source code shall include inline comments for complex logic and comprehensive API documentation.

\paragraph{NFR-24: Modular Architecture}
The system shall be designed with modular architecture allowing individual components to be updated without affecting others.

\paragraph{NFR-25: Version Control}
All source code shall be maintained in Git version control system with meaningful commit messages.

\paragraph{NFR-26: Database Maintenance}
The system shall support routine maintenance operations.

\subsection{Scalability Requirements}

\paragraph{NFR-27: Cattle Population Support}
The system shall support monitoring of at least 5,000 individual cattle per installation without hardware upgrades.

\paragraph{NFR-28: Horizontal Scalability}
The system architecture shall support horizontal scaling of application servers and database instances to accommodate growth.

\paragraph{NFR-29: Data Storage Capacity}
The system shall store complete location and activity data for all cattle for minimum 2 years with efficient retrieval.

\subsection{Portability Requirements}

\paragraph{NFR-30: Android Platform Support}
The mobile application shall run on Android devices.

\paragraph{NFR-31: Cloud Platform Independence}
The system architecture shall minimize dependencies on specific cloud provider features to facilitate potential migration.

\subsection{Availability Requirements}

\paragraph{NFR-32: Planned Maintenance}
Planned maintenance windows shall be scheduled during low-usage periods (midnight to 4 AM) with 24-hour advance notification.

\paragraph{NFR-33: Offline Capability}
The mobile application shall cache recently viewed data and provide read-only access when network connectivity is unavailable.

\section{External Interface Requirements}

\subsection{User Interface Requirements}

\paragraph{IR-01: Mobile Platform Support}
The system shall provide native mobile application for Android.

\paragraph{IR-02: Mobile Navigation Structure}
The mobile application shall implement bottom navigation with sections: Dashboard, Map, Alerts, Cattle List, and Settings.

\paragraph{IR-03: Touch-Friendly Interface}
All interactive elements shall have minimum touch target size of 48x48 dp for accessibility.

\paragraph{IR-04: Responsive Map Interface}
The mobile map interface shall support pinch-to-zoom, two-finger rotation, marker clustering, and smooth panning.

\paragraph{IR-05: Web Browser Compatibility}
The web application shall function correctly on current versions of Chrome, Firefox, Safari, and Edge browsers.

\paragraph{IR-06: Responsive Web Design}
The web interface shall adapt to different screen sizes maintaining usability from 1024x768 to 4K displays.

\subsection{Hardware Interface Requirements}

\paragraph{IR-07: Device Communication Protocol}
The system shall communicate with GPS collar devices using MQTT (Message Queuing Telemetry Transport) protocol over secure TLS (Transport Layer Security) connection.

\paragraph{IR-08: Data Packet Format}
GPS collar devices shall transmit data in JSON format containing device ID, timestamp, GPS coordinates, accuracy, battery percentage, and activity metrics.

\paragraph{IR-09: Mobile Device GPS Integration}
The mobile application may access device GPS to show user location on maps relative to cattle positions.

\subsection{Software Interface Requirements}

\paragraph{IR-10: Cloud IoT Service}
The system shall interface with Azure IoT Hub or AWS IoT Core using standard MQTT or AMQP protocols.

\paragraph{IR-11: Cloud Database Interface}
The system shall interface with cloud database services using standard SQL queries and ORM frameworks.

\paragraph{IR-12: Cloud Storage Interface}
The system shall interface with cloud storage for storing images, reports, and backup files.

\paragraph{IR-13: Mapping Service Integration}
The system shall integrate with Google Maps API or equivalent for map display and geocoding functions.

\paragraph{IR-14: Power BI Integration}
The system shall expose data through REST API endpoints compatible with Power BI data connectors.

\subsection{Communication Interface Requirements}

\paragraph{IR-15: HTTPS Communication}
All communication between client applications and cloud services shall use HTTPS with TLS.

\paragraph{IR-16: RESTful API}
The system shall expose RESTful API following standard HTTP methods (GET, POST, PUT, DELETE).

\paragraph{IR-17: JSON Data Format}
All API communications shall use JSON format for request and response bodies.

\paragraph{IR-18: API Authentication}
API requests shall be authenticated using bearer tokens (JWT) transmitted in Authorization headers.

\section{Design Constraints}

\paragraph{DR-01: IEEE Standards Compliance}
Software development shall adhere to IEEE Std 830-1998 for requirements specification.

\paragraph{DR-02: Android Development Standards}
The mobile application shall follow Android development best practices and Material Design guidelines.

\paragraph{DR-03: Cloud Platform Selection}
The system shall be deployed on either Microsoft Azure or Amazon Web Services cloud infrastructure.

\paragraph{DR-04: Programming Language}
The Android mobile application shall be developed using JavaScript programming language.

\paragraph{DR-05: Database Technology}
The system shall use SQL database (Azure SQL or Amazon RDS) for structured data storage.

\end{spacing}

% ======================================================
\chapter{System Models}
\begin{spacing}{1.5}

\section{Use Case Model}

\subsection{Use Case Diagram}
The use case diagram illustrates the functional interactions between system actors and the CattleCare Pro system. The primary actors are:

\begin{itemize}
\item \textbf{Farm Owner/Administrator:} Interacts with the system to monitor cattle, receive alerts, view reports, and configure system settings
\item \textbf{System Administrator:} Manages user accounts, devices, and system configuration
\item \textbf{GPS Collar Device:} Provides location and activity data (represented as external system)
\end{itemize}

\subsection{Primary Use Cases}

\subsubsection{UC-01: User Login}
\textbf{Actor:} Farm Owner/Administrator, System Administrator

\textbf{Precondition:} User has valid account credentials

\textbf{Main Flow:}
\begin{enumerate}
\item User enters username and password
\item System validates credentials
\item System creates secure session
\item System displays main dashboard
\end{enumerate}

\textbf{Postcondition:} User is authenticated and can access authorized features

\subsubsection{UC-02: Monitor Cattle Location}
\textbf{Actor:} Farm Owner/Administrator

\textbf{Precondition:} User is logged in, GPS collars are active

\textbf{Main Flow:}
\begin{enumerate}
\item User navigates to Map view
\item System retrieves current GPS coordinates
\item System displays cattle locations on interactive map
\item User can zoom, pan, and select individual cattle
\item System shows detailed information for selected cattle
\end{enumerate}

\textbf{Postcondition:} User views current cattle locations

\subsubsection{UC-03: View Activity Data}
\textbf{Actor:} Farm Owner/Administrator

\textbf{Precondition:} User is logged in, activity data is available

\textbf{Main Flow:}
\begin{enumerate}
\item User selects specific cattle or group
\item User specifies date range
\item System retrieves activity data
\item System displays activity graphs and metrics
\item User can compare with historical baselines
\end{enumerate}

\textbf{Postcondition:} User views activity analysis

\subsubsection{UC-04: Receive Alert Notification}
\textbf{Actor:} Farm Owner/Administrator

\textbf{Precondition:} Alert condition is detected

\textbf{Main Flow:}
\begin{enumerate}
\item System detects anomaly (geofence violation, inactivity, etc.)
\item System generates alert record
\item System sends push notification to mobile device
\item User receives and views notification
\item User acknowledges or resolves alert
\end{enumerate}

\textbf{Postcondition:} User is notified of critical event

\subsubsection{UC-05: Generate Report}
\textbf{Actor:} Farm Owner/Administrator

\textbf{Precondition:} User is logged in, data is available

\textbf{Main Flow:}
\begin{enumerate}
\item User navigates to Reports section
\item User selects report type and parameters
\item System generates report
\item System displays report with visualizations
\item User can export report in desired format
\end{enumerate}

\textbf{Postcondition:} User obtains analytical report

\subsubsection{UC-06: Configure Geofence}
\textbf{Actor:} Farm Owner/Administrator

\textbf{Precondition:} User is logged in with admin privileges

\textbf{Main Flow:}
\begin{enumerate}
\item User navigates to Geofence Management
\item User creates new geofence by drawing on map
\item User names and describes geofence
\item System saves geofence configuration
\item System begins monitoring cattle against new boundary
\end{enumerate}

\textbf{Postcondition:} New geofence is active

\subsubsection{UC-07: Manage User Accounts}
\textbf{Actor:} System Administrator

\textbf{Precondition:} Admin is logged in

\textbf{Main Flow:}
\begin{enumerate}
\item Admin navigates to User Management
\item Admin creates, modifies, or deactivates user accounts
\item Admin assigns roles and permissions
\item System validates and saves changes
\item System logs administrative action
\end{enumerate}

\textbf{Postcondition:} User accounts are updated
\subsection{Use Case Diagram} \begin{figure}[H] \centering \includegraphics[width=0.8\textwidth]{use_case.jpg} \caption{Use Case Diagram of CattleCare Pro} \end{figure}

\section{Context Diagram}

The context diagram represents the CattleCare Pro system as a single process and shows its interactions with external entities. The system boundary encompasses all internal processing, while external entities include:

\begin{itemize}
\item \textbf{Farm Owner/Admin:} Submits monitoring requests, receives cattle data and alerts
\item \textbf{System Administrator:} Manages users, devices, and system configuration
\item \textbf{GPS Collar Devices:} Provide location and activity data to system
\item \textbf{Cloud Server:} Stores and processes data
\item \textbf{Power BI:} Receives data for analytics visualization
\end{itemize} 
\subsection{Context Diagram} \begin{figure}[H] \centering \includegraphics[width=0.8\textwidth]{context_diagram.jpg} \caption{Context Diagram of CattleCare Pro} \end{figure}
\end{spacing}

% ======================================================
\chapter{Appendices}
\begin{spacing}{1.5}

\section{Glossary}

\subsection{Technical Terms}

\textbf{Accelerometer:} A sensor that measures acceleration forces, used in collar devices to detect cattle movement and activity levels.

\textbf{API (Application Programming Interface):} A set of protocols and tools for building software applications that specify how components should interact.

\textbf{Baseline Behavior:} The normal or typical activity pattern established for an individual animal over time, used as reference for anomaly detection.

\textbf{Bearer Token:} An authentication token that grants access to protected resources, typically passed in the Authorization header of HTTP requests.

\textbf{Cloud Computing:} Delivery of computing services including servers, storage, databases, and software over the internet.

\textbf{Collar Device:} IoT hardware worn by cattle that includes GPS, accelerometer, battery, and cellular communication capabilities.

\textbf{Dashboard:} A visual display of important information organized for quick monitoring and decision-making.

\textbf{Data Encryption:} Process of converting information into code to prevent unauthorized access.

\textbf{Geofencing:} Use of GPS technology to create virtual geographic boundary, enabling software to trigger response when device enters or leaves area.

\textbf{IoT Hub:} Cloud service that provides bidirectional communication between IoT application and devices it manages.

\textbf{JSON (JavaScript Object Notation):} Lightweight data interchange format that is easy for humans to read and write.

\textbf{JWT (JSON Web Token):} Compact means of representing claims to be transferred between two parties, commonly used for authentication.

\textbf{Power BI:} Business analytics service by Microsoft providing interactive visualizations with self-service business intelligence capabilities.

\textbf{Push Notification:} Message sent from application to user's device that appears even when application is not actively being used.

\section{Analysis Models}

\subsection{Cost-Benefit Analysis}

\subsubsection{Implementation Costs}
\begin{itemize}
\item GPS collar devices: \$50-100 per device
\item Cloud hosting: \$200-500 per month
\item Mobile app development: One-time development cost
\item System maintenance: Ongoing operational cost
\end{itemize}

\subsubsection{Expected Benefits}
\begin{itemize}
\item Reduced cattle loss from theft (estimated 5-10\% reduction)
\item Early disease detection reducing mortality
\item Improved operational efficiency (20-30\% time savings)
\item Better grazing management
\item Data-driven decision making
\end{itemize}

\subsection{Risk Analysis}

\begin{table}[H]
\centering
\caption{Risk Assessment Matrix}
\begin{tabular}{|p{4cm}|p{2cm}|p{2cm}|p{5cm}|}
\hline
\textbf{Risk} & \textbf{Probability} & \textbf{Impact} & \textbf{Mitigation Strategy} \\ \hline
Network coverage gaps & High & High & Store data locally, sync when connected \\ \hline
Device battery failure & Medium & Medium & Low battery alerts, regular maintenance \\ \hline
Cloud service outage & Low & High & Redundant systems, local caching \\ \hline
User adoption resistance & Medium & Medium & Training programs, gradual rollout \\ \hline
Data security breach & Low & High & Encryption, access control, auditing \\ \hline
GPS signal loss & Medium & Medium & Multiple position sources, interpolation \\ \hline
\end{tabular}
\end{table}

\section{Future Enhancements}

\subsection{Planned Features for Version 2.0}
\begin{itemize}
\item \textbf{Health Monitoring:} Integration with temperature and heart rate sensors for comprehensive health tracking
\item \textbf{Breeding Management:} Automated estrus detection and breeding recommendations
\item \textbf{Feed Optimization:} Integration with feeding systems for automated nutrition management
\item \textbf{Weather Integration:} Correlation of weather data with cattle behavior
\item \textbf{Multi-Farm Support:} Capability to manage multiple farm locations from single interface
\item \textbf{Veterinary Integration:} Direct sharing of data with veterinary systems
\item \textbf{Machine Learning:} Advanced predictive analytics for disease outbreak prediction
\item \textbf{Voice Commands:} Hands-free operation using voice interface
\item \textbf{AR Visualization:} Augmented reality features for field identification
\end{itemize}

\subsection{Technology Roadmap}
\begin{itemize}
\item \textbf{Q1 2025:} iOS application development
\item \textbf{Q2 2025:} Advanced analytics and ML integration
\item \textbf{Q3 2025:} Health sensor integration
\item \textbf{Q4 2025:} Multi-farm and enterprise features
\end{itemize}

\section{Supporting Documentation References}

\subsection{Technical Documentation}
\begin{itemize}
\item Android Development Guide: https://developer.android.com/guide
\item Kotlin Programming Language: https://kotlinlang.org/docs
\item Google Maps Android API: https://developers.google.com/maps/documentation/android-sdk
\item Firebase Cloud Messaging: https://firebase.google.com/docs/cloud-messaging
\item Power BI Developer Documentation: https://docs.microsoft.com/power-bi/developer
\end{itemize}

\subsection{Academic References}
\begin{itemize}
\item Wathes, C. M., et al. (2008). "Precision livestock farming." Journal of Agricultural Science.
\item Rutten, C. J., et al. (2013). "Invited review: Sensors to support health management on dairy farms." Journal of Dairy Science.
\item Berckmans, D. (2014). "Precision livestock farming technologies for welfare management in intensive livestock systems."
\end{itemize}
\end{spacing}

\end{document}